\documentclass{article}

\title{Income and Consumption Inequality in Taiwan, 1981 - 2023}
\author{Bo-Yan Huang\footnote{Department of Economics, National Taiwan University, Taipei, Taiwan}}
\date{June 2025}

% Bring in setting.tex
\usepackage[a4paper,top=2cm,bottom=2cm,left=3cm,right=3cm,marginparwidth=1.75cm]{geometry}
\usepackage{graphicx}
\usepackage[colorlinks=true, allcolors=blue]{hyperref}
\usepackage{setspace}
\usepackage{subcaption}
\usepackage{mwe}

\setstretch{1.3}
\setlength\parindent{2em}

\begin{document}

\maketitle

\section{Introduction}
\label{sec:introduction}

In Taiwan, the evolution of economic inequality is often overshadowed by other topics in the field of economics.
However, understanding the dynamics of inequality is crucial for comprehending the broader economic landscape. 
Most studies on inequality in Taiwan have focused on income distribution and income gap, with less attention given to consumption inequality.
For instance, Chang, Lin, and Lee (\citeyear{TW_classes_inc}) analyzed the income distribution in Taiwan from 1990 to 2020, highlighting the increase in income and education inequality during this period.
The authors apply the Age-Period-Cohort (APC) model to reveal that classes and education inequality primarily drive the increase in income inequality, and the generational gap between the baby boomers and millennials does not decrease over time. 
At the same time, Chang (\citeyear{TW_party_inc_ineq}) focused on the impact of government redistribution and political parties' actions.
The author points out that the Democratic Progressive Party (DPP) share in the Legislative Yuan is negatively correlated with income inequality. However, a president from the DPP is negatively correlated with redistribution, which could worsen income inequality in the long run.
Both studies use the Survey of Family Income and Expenditure (SFIE) as their primary data set. This repeated cross-sectional survey, started in 1978 and conducted yearly, provides detailed information on Taiwan's household income and expenditure patterns.

Consumption inequality has been less frequently studied in Taiwan. Hong (\citeyear{TW_consumption_inequality}) used the SFIE from 1980 to 2007 to analyze the evolution of consumption inequality in Taiwan, focusing on the Gini coefficient and the dynamics of consumption percentiles. 
The study found that consumption inequality decreased after 1996 due to the redistribution effect of public health insurance in Taiwan and the growing elderly population in our demographic structure.
However, the decrease in consumption inequality could be a double-edged sword since the other reason for the decline is the low consumption growth rate of wealthy families, which could lead to an economy that stagnates.

One reason for the lack of attention to consumption in Taiwan and other countries is the difficulty in measuring consumption accurately.
First, unlike income, which can be directly observed through tax records, consumption is often inferred from expenditure data that the government does not document.
Therefore, most consumption data are often collected through surveys, which have a smaller sample size and are more prone to measurement errors and biases.
Currently, the SFIE is the only survey that provides detailed information on household consumption in Taiwan, while The Consumer Expenditure Survey (CE) is the only one in the US (Attanasio \& Pistaferri, \citeyear{JEP_Consumption_Inequality}).

Second, the transformation from expenditure to consumption is not straightforward. Surveys like the SFIE typically report household expenditures, which may not accurately reflect households' actual consumption.
For example, households may purchase goods in bulk or invest in durable goods, which can lead to discrepancies between expenditure and consumption. Most surveys lack information on the stock of durable goods and their current value.
Another example is that consumption can be produced at home using time and goods, like childcare and home-cooked meals. Also, households may have different preferences for leisure and consumption, which can lead to differences in how they allocate their resources.
The most challenging problem to overcome is the assumption that all households face the same prices for goods and services. This assumption violates the nature of durable goods such as housing and automobiles, which can have enormous price variations. Ignoring these differences can lead to an inaccurate representation of consumption patterns and inequality.

This paper uses SFID data to comprehensively analyze income and consumption inequality in Taiwan from 1981 to 2023 to fill in the missing pieces of inequality dynamics after 2008. This is interesting because economic growth in Taiwan has slowed since the 2008 financial crisis, and the COVID-19 pandemic has further exacerbated the situation.
The income distribution between households was stable before 1994, and the Gini coefficient for individual earnings stabilized at around 0.30 due to many factors (Bourguignon, Fournier, \& Gurgand, \citeyear{TW_stable_dist}).
However, the Gini coefficient for household earnings has increased since 1994, reaching 0.35 in 2023, as shown in the top right panel of Figure \ref{fig:Indi_to_HH}.
This suggests that economic inequality will become a problem after economic growth slows down, even in a country with steady income distribution previously, like Taiwan.
The paper is organized as follows: Section \ref{sec:data} describes the data used in the analysis, Section \ref{sec:household_inequality} presents the household-level inequality measures, including earnings, consumption, and their cyclical dynamics, and Section \ref{sec:conclusion} concludes the paper.

\section{Data}
\label{sec:data}

% \subsection{Survey of Family Income and Expenditure (SFIE)}
\subsection{Survey of Family Income and Expenditure (SFIE)}

The data used in this paper is the \citelink{DGBAS_SFIE}{SFIE} from 1981 to 2023, which is a repeated cross-sectional survey conducted annually since 1978 by the Directorate-General of Budget, Accounting and Statistics (\citeauthor{DGBAS_SFIE}) in Taiwan.
The survey provides a rich dataset for analyzing income and expenditure patterns, including earnings, private transfers, asset income, government transfers, and consumption expenditures.
The survey samples around 16,000 households each year; however, the NA rate of specific columns, such as labor earnings and industry earnings shown in Figure \ref{fig:missing}, is increasing across years.
\begin{figure}
    \centering
    \begin{subfigure}[t]{0.475\textwidth}
        \centering
        \includegraphics[width=\textwidth]{figures/missing/itm190.png}
        \label{fig:missing_labor}
    \end{subfigure}
    \begin{subfigure}[t]{0.475\textwidth}
        \centering
        \includegraphics[width=\textwidth]{figures/missing/itm240.png}
        \label{fig:missing_industry}
    \end{subfigure}
    \caption{Missing Rate of Labor Earnings and Industry Earnings in SFIE}
    \label{fig:missing}
\end{figure}
In the SFIE, both zero and no response are treated as missing values, which can lead to an underestimation of income and consumption inequality if we treat all missing values as zero.
Therefore, the analysis in this paper will focus on households with positive earnings and consumption expenditures aligned with the design proposed by Heathcote, Perri, \& Violante (\citeyear{HEATHCOTE_2010}).
I treat the missing values as zero when doing addition and subtraction, such as calculating household earnings as the sum of labor and industry earnings.
Note that there is a missing value in 2012 of Figure \ref{fig:missing}, this is because the data file of SFIE for 2012 has all the demographic information, but the income and expenditure data are ALL missing values.
Therefore, I exclude the 2012 data from the analysis.

\subsection{CPI adjustment}

The SFIE provides nominal values for income and expenditures, which need to be adjusted for inflation to obtain real values. 
The price deflator used in this paper is the Consumer Price Index (\citelink{DGBAS_CPI}{CPI}) provided by the \citeauthor{DGBAS_CPI}.
Figure \ref{fig:CPI} shows the CPI in Taiwan from 1981 to 2023 normalized to 100 in 2021.
\begin{figure}
    \centering
    \includegraphics[width=0.475\textwidth]{figures/cpi.png}
    \caption{Consumer Price Index (CPI) in Taiwan, 1981 - 2023}
    \label{fig:CPI}
\end{figure}

\subsection{Equivalize Household Income and Consumption}

To compare households of different sizes and compositions, I use the OECD equivalence scale to adjust household income and consumption to per-adult-equivalent levels. The OECD scale assigns a weight of 1 to the first adult, 0.7 to each additional adult, and 0.5 to each child. For the definition of child, I define a child as age 17 or younger, the original OECD definition is age 13 or younger.

\section{Household-level Inequality}
\label{sec:household_inequality}

Here, I present the household-level inequality measures, including earnings and consumption inequality, and their cyclical dynamics. This paper skips individual-level inequality, as it is more suitable to estimate using tax records. Individual-level consumption is not available in the SFIE, as it only recordes household-level non-durable consumption.
Estimating consumption inequality using household-level data is also more suitable, as households can share consumption goods and services, such as housing and food, which are not easily separable at the individual level.
In this section, I will follow the structure of Heathcote, Perri, \& Violante (\citeyear{HEATHCOTE_2010}) to construct a complete picture of household-level inequality from basic earnings to disposable income and consumption. 


\subsection{Inequality in Earnings}

\begin{figure}
    \centering
    \begin{subfigure}[t]{0.475\textwidth}
        \centering
        \includegraphics[width=\textwidth]{figures/Fig_1/Fig_1a_Var_Earnings.png}
        \label{fig:earnings_inequality_Var}
    \end{subfigure}
    \begin{subfigure}[t]{0.475\textwidth}
        \centering
        \includegraphics[width=\textwidth]{figures/Fig_1/Fig_1b_Gini_Earnings.png}
        \label{fig:earnings_inequality_Gini}
    \end{subfigure}
    \begin{subfigure}[t]{0.475\textwidth}
        \centering
        \includegraphics[width=\textwidth]{figures/Fig_1/Fig_1c_P50_P10_Earnings.png}
        \label{fig:earnings_inequality_P50_P10}
    \end{subfigure}
    \begin{subfigure}[t]{0.475\textwidth}
        \centering
        \includegraphics[width=\textwidth]{figures/Fig_1/Fig_1d_P90_P50_Earnings.png}
        \label{fig:earnings_inequality_P90_P50}
    \end{subfigure}
    \caption{Various Measures of Household Earnings Inequality}
    \label{fig:earnings_inequality}
\end{figure}

\begin{figure}
    \centering
    \includegraphics[width=0.95\textwidth]{figures/Fig_2/Fig_2_percentiles_a1995.png}
    \caption{Percentiles of the Household Earnings Distribution}
    \label{fig:earnings_inequality_cyclic}
\end{figure}

Figure \ref{fig:earnings_inequality} plots four measures of dispersion in household earnings: variance of log, Gini coefficient, P50-P10 ratio, and P90-P50 ratio.
Though all measures of dispersion have increased since 1981, there are some minor differences in their dynamics.
The top right variance and the bottom left P50-P10 ratio track each other extremely closely, both spikes around 2000 Dot-com bubble and remain relatively steady after that.
The Gini coefficient in the top right panel shows a steady increase from 0.29 in 1981 to 0.36 in 2023, indicating that earnings inequality has risen over the past four decades.
The bottom right P90-P50 ratio is the most interesting one, it also shows a spike around 2000, but it has been decreasing since then, indicating that the earnings of the top 10\% have not been growing as fast as the median earnings after 2000.
It is worth noting that though the dispersion of earnings has been drastically increasing visually since 1981, the growth of these measurements is actually quite modest compared to other countries, such as the US.
The Gini coefficient for household earnings in Taiwan has increased only 7 log points during the past four decades, while Bourguignon, Fournier, \& Gurgand (\citeyear{TW_stable_dist}) showed that in the US, the Gini coefficient for household earnings has increased roughly 30 log points from 1968 to 2005.



\subsection{Cyclical Dynamics of Earnings Inequality}

Figure \ref{fig:earnings_inequality_cyclic} shows the trend in percentiles at different points in the distribution of household earnings from 1995 to 2023. The log earnings were normalized to 0 in 1995. Therefore, the y-axis shows the percentage change in earnings (as log points) relative to 1995.
Here, I choose to show the result after and normalize it to 1995. The full results from 1981 to 2023 are in the appendix \ref{sec:appendix_cyclical_dynamics} where Figure \ref{fig:appendix_cyclic_earnings_1981} normalized to 1981 and \ref{fig:appendix_cyclic_earnings_1995} normalized to 1995. 
The reason for this choice is that all the percentiles grew over 100 log points from 1981 to 1995; that is, the earnings of all households have tripled since 1981.
This finding coincides with Bourguignon, Fournier, \& Gurgand (\citeyear{TW_stable_dist}) that before 1994, the income of Taiwanese households was in both a fast developing and stable environment.
Therefore, putting years before 1995 in the main figure could dilute the interpretation of recent trends, as the rapid growth phase would dominate the visual impression. By focusing on the period after 1995, we can better highlight the cyclical and structural changes in earnings inequality that occurred after Taiwan's high-growth era ended. This approach allows for a more precise analysis of the dynamics relevant to the current policy and economic environment.


\subsection{From Individual to Household Inequality}

\begin{figure}
    \centering
    \begin{subfigure}[t]{0.475\textwidth}
        \centering
        \includegraphics[width=\textwidth]{figures/Fig_3/Fig_3a_Var_indHH.png}
        \label{fig:Indi_to_HH_Var}
    \end{subfigure}
    \begin{subfigure}[t]{0.475\textwidth}
        \centering
        \includegraphics[width=\textwidth]{figures/Fig_3/Fig_3b_Gini_indHH.png}
        \label{fig:Indi_to_HH_Gini}
    \end{subfigure}
    \begin{subfigure}[t]{0.475\textwidth}
        \centering
        \includegraphics[width=\textwidth]{figures/Fig_3/Fig_3c_Var_single.png}
        \label{fig:Indi_to_HH_Var_single}
    \end{subfigure}
    \begin{subfigure}[t]{0.475\textwidth}
        \centering
        \includegraphics[width=\textwidth]{figures/Fig_3/Fig_3d_married_ratio.png}
        \label{fig:Indi_to_HH_married_ratio}
    \end{subfigure}
    \begin{subfigure}[t]{0.475\textwidth}
        \centering
        \includegraphics[width=\textwidth]{figures/Fig_3/Fig_3e_two_earner_ratio.png}
        \label{fig:Indi_to_HH_two_earner_ratio}
    \end{subfigure}
    \begin{subfigure}[t]{0.475\textwidth}
        \centering
        \includegraphics[width=\textwidth]{figures/Fig_3/Fig_3f_correlation.png}
        \label{fig:Indi_to_HH_correlation}
    \end{subfigure}
    \caption{Understanding the role of the Family for Earnings Inequality}
    \label{fig:Indi_to_HH}
\end{figure}

\subsection{Private Transfers and Asset Income}

\begin{figure}
    \centering
    \begin{subfigure}[t]{0.475\textwidth}
        \centering
        \includegraphics[width=\textwidth]{figures/Fig_4/Fig_4a_Var_inc.png}
        \label{fig:Trans_Asset_Var1}
    \end{subfigure}
    \begin{subfigure}[t]{0.475\textwidth}
        \centering
        \includegraphics[width=\textwidth]{figures/Fig_4/Fig_4b_Gini_inc.png}
        \label{fig:Trans_Asset_Gini1}
    \end{subfigure}
    \begin{subfigure}[t]{0.475\textwidth}
        \centering
        \includegraphics[width=\textwidth]{figures/Fig_4/Fig_4c_Var_inc.png}
        \label{fig:Trans_Asset_Var2}
    \end{subfigure}
    \begin{subfigure}[t]{0.475\textwidth}
        \centering
        \includegraphics[width=\textwidth]{figures/Fig_4/Fig_4d_Gini_inc.png}
        \label{fig:Trans_Asset_Gini2}
    \end{subfigure}
    \caption{From Household Earnings to Pre-Government Income}
    \label{fig:Trans_Asset}
\end{figure}

\subsection{Government Redistribution}

\begin{figure}
    \centering
    \begin{subfigure}[t]{0.475\textwidth}
        \centering
        \includegraphics[width=\textwidth]{figures/Fig_5/Fig_5a_Var_inc.png}
        \label{fig:Gov_Var1}
    \end{subfigure}
    \begin{subfigure}[t]{0.475\textwidth}
        \centering
        \includegraphics[width=\textwidth]{figures/Fig_5/Fig_5b_Gini_inc.png}
        \label{fig:Gov_Gini1}
    \end{subfigure}
    \begin{subfigure}[t]{0.475\textwidth}
        \centering
        \includegraphics[width=\textwidth]{figures/Fig_5/Fig_5c_Var_inc.png}
        \label{fig:Gov_Var2}
    \end{subfigure}
    \begin{subfigure}[t]{0.475\textwidth}
        \centering
        \includegraphics[width=\textwidth]{figures/Fig_5/Fig_5d_Gini_inc.png}
        \label{fig:Gov_Gini2}
    \end{subfigure}
    \caption{From Pre-Government to Disposable Income}
    \label{fig:Gov}
\end{figure}

The result shows that the smoothing effect of the family, such as private transfer, is significant, and the government redistribution also has a significant effect on income inequality.

\subsection{From Income to Consumption Inequality}

\begin{figure}
    \centering
    \begin{subfigure}[t]{0.475\textwidth}
        \centering
        \includegraphics[width=\textwidth]{figures/Fig_6/Fig_6a.png}
        \label{fig:Consumption_Var}
    \end{subfigure}
    \begin{subfigure}[t]{0.475\textwidth}
        \centering
        \includegraphics[width=\textwidth]{figures/Fig_6/Fig_6b.png}
        \label{fig:Consumption_Gini}
    \end{subfigure}
    \begin{subfigure}[t]{0.475\textwidth}
        \centering
        \includegraphics[width=\textwidth]{figures/Fig_6/Fig_6c.png}
        \label{fig:Consumption_P50_P10}
    \end{subfigure}
    \begin{subfigure}[t]{0.475\textwidth}
        \centering
        \includegraphics[width=\textwidth]{figures/Fig_6/Fig_6d.png}
        \label{fig:Consumption_P90_P50}
    \end{subfigure}
    \caption{From Disposable Income to Consumption}
    \label{fig:Consumption}
\end{figure}

\subsection{Cyclical Dynamics of Consumption Inequality}

\begin{figure}
    \centering
    \includegraphics[width=0.95\textwidth]{figures/Fig_7/Fig_7_percentiles_a1995.png}
    \caption{Percentiles of the Household Consumption Distribution}
    \label{fig:Consumption_cyclic}
\end{figure}

\subsection{Food Consumption and Housing Services}

\begin{figure}
    \centering
    \begin{subfigure}[t]{0.475\textwidth}
        \centering
        \includegraphics[width=\textwidth]{figures/Fig_8/Fig_8a_Var_Food.png}
        \label{fig:Food_Var}
    \end{subfigure}
    \begin{subfigure}[t]{0.475\textwidth}
        \centering
        \includegraphics[width=\textwidth]{figures/Fig_8/Fig_8b_Gini_Food.png}
        \label{fig:Food_Gini}
    \end{subfigure}
    \begin{subfigure}[t]{0.475\textwidth}
        \centering
        \includegraphics[width=\textwidth]{figures/Fig_8/Fig_8c_Var_Housing.png}
        \label{fig:Housing_Var}
    \end{subfigure}
    \begin{subfigure}[t]{0.475\textwidth}
        \centering
        \includegraphics[width=\textwidth]{figures/Fig_8/Fig_8d_Gini_Housing.png}
        \label{fig:Housing_Gini}
    \end{subfigure}
    \caption{Food and Housing Consumption Inequality}
    \label{fig:Food_Housing}
\end{figure}


\section{Conclusion}
\label{sec:conclusion}

\newpage

% Format for bibliography
\DeclareFieldFormat[misc]{title}{{#1}}
% Format for citations
\DeclareFieldFormat[misc]{citetitle}{{#1}}  

\printbibliography

\newpage
% Appendices
\begin{appendices}
\section{Extra Figure for Cyclical Dynamics}
\label{sec:appendix_cyclical_dynamics}

\vspace{2em}

\begin{figure}[h]
    \centering
    \includegraphics[width=0.88\textwidth]{figures/Fig_2/Fig_2_percentiles_1981.png}
    \caption{Percentiles of the Household Earnings Distribution, 1981 - 2023}
    \label{fig:appendix_cyclic_earnings_1981}
\end{figure}

\vspace{4em} 

\begin{figure}[h]
    \centering
    \includegraphics[width=0.88\textwidth]{figures/Fig_2/Fig_2_percentiles_1995.png}
    \caption{Percentiles of the Household Earnings Distribution, 1981 - 2023}
    \label{fig:appendix_cyclic_earnings_1995}
\end{figure}

\begin{figure}[h]
    \centering
    \includegraphics[width=0.88\textwidth]{figures/Fig_7/Fig_7_percentiles_1981.png}
    \caption{Percentiles of the Household Consumption Distribution, 1981 - 2023}
    \label{fig:appendix_cyclic_consumption_1981}
\end{figure}

\begin{figure}[h]
    \centering
    \includegraphics[width=0.88\textwidth]{figures/Fig_7/Fig_7_percentiles_1995.png}
    \caption{Percentiles of the Household Consumption Distribution, 1981 - 2023}
    \label{fig:appendix_cyclic_consumption_1995}
\end{figure}

\end{appendices}

\end{document}